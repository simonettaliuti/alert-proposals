\chapter{Experimental Setting}
\label{chap:setting}


For the proposed measurements, we require a large kinematic coverage for the scattered electrons and the ability to tag low momentum spectators. The CLAS12 detector with 11~GeV electron beam is well suitable to access the valence region. A low energy recoil detector with good performance is the key parameter for the success of such measurements.

\section{CLAS12 Forward Detector}
The CLAS12 detector is designed to operate with 11 GeV beam at a electron-nucleon luminosity of $L = 1\times10^{35}$cm$^{-2}$s$^{-1}$. The baseline configuration of the CLAS12 detector consists of the Forward Detector and the Central Detector packages~\cite{CD} (see Fig.~\ref{fig:fd}).

\begin{figure}
  \begin{center}
    \includegraphics[angle=0, width=0.75\textwidth]{./../Detector/fig-chap2/fd}
    \caption{The schematic layout of the CLAS12 baseline design.}
    \label{fig:fd}
  \end{center}
\end{figure}

The scattered electrons will be detected in the forward detector which consists of the High Threshold Cherenkov Counters (HTCC), Drift Chambers (DC), the Low Threshold Cherenkov Counters (LTCC), the Time-of-Flight scintillators (TOF), the Forward Calorimeter and the Preshower Calorimeter. The charged particle identification in the forward detector is achieved by utilizing the combination of the HTCC, LTCC and TOF arrays with the tracking information from the Drift Chambers. The HTCC together with the Forward Calorimeter and the Preshower Calorimeter will provide pion rejection factor of more than 2000 up to momentum of 4.9 GeV, and a rejection factor of 100 above 4.9 GeV.

\section{Low Energy Recoil Detector}
In addition to the Forward Detector package, we require a low energy recoil detector which has adequate momentum and spatial resolution, and good particle identification for recoiling charged particles (p, $^2$H, $^3$H and $^3$He). We investigate the feasibility of using the CLAS12 Central Detector and the BoNuS Detector~\cite{bonus6,bonus12}. As those seem not suitable for the proposed measurement we propose a new detector for our measurement, that could also be suitable for the BoNuS12 experiment~\cite{bonus12}.

\subsection{Central Detector}
CLAS12 Central Detector~\cite{CD} is designed to detect various charged particles over a wide momentum and angular range. The main detector package includes:
\begin{itemize}
\item Solenoid Magnet: provides a central longitudinal magnetic field up to 5 Tesla, serves to curl emitted low energy M{\o}ller electrons and determine particle momenta through tracking.
\item Central Tracker: consists of 3 layers of silicon strips and 3 layers of Micromegas. The thickness of a single silicon layer is $300\mu$m~\cite{SVT}.
\item Central Time-of-Flight: an array of scintillator paddles with a cylindrical geometry of radius 26 cm and length 50 cm, and the thickness of the detector is 2 cm with designed timing resolution of $\sigma_t = 50$ ps, used to separate pions and protons up to 1.2 GeV/$c$.
\item Central Neutron Detector:  three radial layers of 3 cm thick scintillator bars arranged cylindrically to identify neutrons above 200 MeV/c up to 1.2 GeV/c with 10\% momentum resolution.
\end{itemize}

However, the current design is not optimal for low energy particles ($p<300$ MeV/$c$) due to the energy loss in the first 2 silicon strips. The momentum detection threshold is $\sim 200$ MeV for protons, $\sim 350$ MeV for deuterons and even higher for $^3$H and $^3$He. These values are significantly too large for our proposed measurement, therefore other options should be explored.

\subsection{BoNuS 12 Radial Time Projection Chamber}
The original BoNuS detector was built for Hall B experiment E03-012 to study the neutron structure by scattering electrons off an almost on-shell neutron in the deuterium target. The purpose of the detector was to tag the resulting low energy protons ($p>60$ MeV/$c$). The key component for detecting the slow protons was the Radial Time-projection Chamber (RTPC) based on Gas Electron Multipliers (GEM). The second run period (EG6) used a $^4$He gas target and an improved RTPC to detect recoiling $\alpha$ particles in coherent scattering. The major improvements of the EG6 RTPC were full cylindrical coverage and higher data taking rate.

The approved 12 GeV BoNuS (BoNuS12) proposal is planning to use a similar device with some upgrades. The target gas cell length will be doubled, and the new RTPC will be longer as well, leading to a doubling in luminosity and an increased acceptance. Taking advantage of the larger bore ($\sim 700$ mm) of the CLAS12 5-Tesla solenoid magnet, the maximum radial drift length will be increased from the present 3 cm to 6 cm, improving the momentum resolution by 50\%~\cite{bonus12} and extending the momentum coverage. The main features of the proposed BoNuS12 detector are summarized in Table~\ref{tab:bonus}.

\begin{table}
\caption{\label{tab:bonus}Main features of the BoNuS 12 detector.}
\begin{tabular}{|c|c|}
\hline
Radial drift length & 6 cm\\
\hline
longitudinal length & $\sim$ 40 cm\\
\hline
Gas mixture & 80\% helium/20\% DME\\
\hline
Azimuthal coverage & 2$\pi$\\
\hline
Momentum range & 70-250 MeV/$c$\\
\hline
DAQ rate & $\sim 3$kHz\\
\hline
Solenoidal field & $\sim 5$ T\\
\hline
Gas target length/diameter & $\sim$40 cm/6 mm\\
\hline
Target wall & 25 $\mu$m Kapton\\
\hline
Target window & 15 $\mu$m Al\\
\hline
Target pressure & 7.5 atm\\
\hline
\end{tabular}
\end{table}

The results from the 2005 BoNuS and 2009 EG6 run demonstrate the successful performance of the RTPC.
The agreement between the vertex trajectory for a recoiling $^4$He and the electron in CLAS is shown in Fig.~\ref{fig:EG6elastic_delta} from EG6.  

\begin{figure}
  \begin{center}
    \includegraphics[angle=0, width=0.32\textwidth]{./../Detector/fig-chap2/eg6elastic_dz_small}
    \includegraphics[angle=0, width=0.32\textwidth]{./../Detector/fig-chap2/eg6elastic_dthe_small}
    \includegraphics[angle=0, width=0.32\textwidth]{./../Detector/fig-chap2/eg6elastic_dphi_small}
    \caption{Difference between the reconstructed vertex trajectory ($z$,$\theta$,$\phi$) measured by EG6's RTPC and that calculated from CLAS's electron for $^4$He data at 1.2 GeV.\label{fig:EG6elastic_delta}}
  \end{center}
\end{figure}

In principle, particle identification can be obtained from the RTPC through the energy loss $dE/dx$ in the detector as a function of the particle momentum (see Fig.~\ref{fig:eloss}). However, this necessitates gain calibration, which according to the BoNuS study, varies a lot over the surface of the detector. This is likely due to the GEM gain variations across different regions arising from slight misalignments and wrinkles. The upgraded EG6 detector also exhibits this characteristic, shown in Fig.~\ref{fig:EG6gains}. Yet, even with perfect calibration, because we actually measure the curvature $\propto p/Z$ and not the momentum, $^3$H and $^3$He curves are very close (see Fig.~\ref{fig:eloss}). Such a small difference is impossible to discriminate on an event by event basis because of the intrinsic width of the $dE/dx$ distributions. This feature is not problematic when using the deuterium target, but makes the RTPC improper for our measurement on the helium 4 target.

Another issue with the RTPC is its slow response time due to the drift time ($\sim 5\mu$s). Indeed, a much faster recoil detector could be included in the trigger and have significant impact on background rejection. Since data acquisition speed was the main limiting factor for both BoNuS and EG6 runs in CLAS, including the recoil detector in the trigger would allow to run at higher luminosities. Indeed events without a hit in the recoil detector would not be recorded and reduce the trigger frequency significantly.

\begin{figure}
  \begin{center}
    \includegraphics[angle=0, width=0.5\textwidth]{./../Detector/fig-chap2/pz}
    \caption{Calculation of energy loss in Neon gas as a function of the particle momentum divided by its charge for different nuclei. }
    \label{fig:eloss}
  \end{center}
\end{figure}

\begin{figure}
  \begin{center}
    \includegraphics[angle=0, width=0.8\textwidth]{./../Detector/fig-chap2/eg6_1stPassGains}
    \caption{Gain calibration for the 3200 readout pads of EG6's RTPC measured from 1.2 GeV $^4$He elastic events.  Missing regions are due to low statistics or broken channels.}
    \label{fig:EG6gains}
  \end{center}
\end{figure}

\subsection{New Design}
The limitations of the RTPC technology made necessary the development of another option to access the physical observables described before. We propose for our experiment a new low energy recoil detector design, described in the next section, that would provide good timing and energy loss information and a total energy measurement for each track. The fast timing will allow a tight time coincidence with CLAS12, thereby reducing the background that was encountered in RTPC detectors. It will also permit to include the recoil detector in the data acquisition trigger, this will largely reduce triggering on events from the target windows, which are outside the acceptance and events with recoil too slow to exit the target that cannot be used in the analysis. 

Finally, the use of a time of flight and total energy measurements will provide improved particle identification for the recoiling nuclei without ambiguity for $^3$H and $^3$He. The features and requirements for this new detector are compared with the current RTPC design for BoNuS~12 in Table~\ref{tab:comp}. The transverse momentum and $z$ resolution are given by the BoNuS specifications.

\begin{table}[ht!]
\caption{\label{tab:comp}Comparison between the RTPC (left column) and the new tracker (right column).}
\begin{tabular}{|c|c|c|}
\hline
\textbf{Detectors}  & \textbf{RTPC}        & \textbf{New Tracker}\\
\hline
Drift region radius & 14 cm                & 9 cm\\
\hline
Longitudinal length & $\sim$ 40 cm         & $\sim$ 40 cm \\
\hline
Gas mixture         & 80\% helium/20\% DME & 90\% helium/10\% isobutane \\
\hline
Azimuthal coverage  & 2$\pi$               & 2$\pi$\\
\hline
Momentum range      & 70-250 MeV/$c$ for protons & 70-250 MeV/$c$ for protons\\
\hline
Transverse momentum resolution & 10\% at 100~MeV/c for protons & 10\% at 100~MeV/c for protons\\
\hline
$z$ resolution & 3~mm & 3~mm \\
\hline
Solenoidal field    & $\sim 5$ T           & $\sim 5$ T \\
\hline
Particles separated & p                    & p, $^2$H, $^3$H, $^3$He, $\alpha$ \\
\hline
Trigger             & can not be included  & can be included \\
\hline
\end{tabular}
\end{table}

\section{Low Energy Recoil Detector}

Our proposed detector is composed of two part, a drift chamber and scintillators. The drift chamber will be composed of 6 layers of sense wires to provide trajectory information. Together with a simple array of scintillators placed inside the chamber after the last wires to reduce the energy threshold. The good time resolution, thus position resolution, of the drift chamber coupled with the scintillators will provide energy loss, timing and azimuthal angle for a large domain of particles and energy.

The drift chamber will be filled with a light gas mixture such as He(90\%) and iC$_4$H$_10$(10\%) so as not to be sensitive to relativistic particles (i.e. electrons, gammas) and neutron backgrounds. It also increases the drift speed of electrons created during the ionisation, which allows the chamber to stand a higher particle rate. The gas should be at atmospheric pressure but studies will be carried out to evaluate the possibility of working at a lower pressure. Based on these characteristics, the signals of this chamber and the scintillators can be used in the CLAS trigger thus reducing the DAQ frequency allowing to increase the luminosity.

\section{Design}

The detector must be designed to fit inside the outermost layer of Micromegas; the silicon vertex tracker and the other layers of Micromegas will be removed. The available space has thus an outer radius of 200 mm.

A schematic layout of the preliminary design is shown in Fig.~\ref{fig:new_lay}. The different elements are all covering $2\pi$ and are 400~mm long with an effort made to reduce the particle energy loss through the materials. It is composed of:
\begin{itemize}
\item a cylindrical target, that compared to the EG6 run, is longer ($\sim \!40$ cm), wider (outer radius is 6~mm) and with lower pressure ($\sim \!3$ atm) in order to use a thinner target wall ($\sim \!15\mu$m Kapton)~\footnote{During EG6 run, the pressure of the drift gas in RTPC was $\sim \!1$ atm, and the pressure of the target was $\sim \!6.5$ atm.};
\item a space filled with helium with an outer radius of 30~mm where the rate of Moller electrons is too high;
\item the drift chamber, up to a radius of 85~mm, to detect the trajectory of the low energy nuclear recoil;
\item an array of segmented plastic scintillators placed inside the gaseous chamber, with total thickness of 50~mm.
\end{itemize}
%\\


\begin{figure}[ht!]
  \begin{center}
    \includegraphics[angle=0, width=0.75\textwidth]{./../Detector/fig-chap2/View_det_names.pdf}
    \caption{The schematic layout of the new recoil detector design, viewed from the beam direction.}
    \label{fig:new_lay}
  \end{center}
\end{figure}

\subsection{Drift chamber}

While drift chambers are very usefull to cover large areas at a moderate price compared with other detectors, huge progress have been made in terms of rapidity to stand higher rates using better electronics, shorter distance between wires and optimization of the electric field over pressure ratio. Our design is based on other chambers developped recently. For example for the dimuon arm of ALICE at CERN, drift chambers with cathode planes were built in Orsay \cite{AliceMuonArmChamber}. The gap between sense wires is 2.1~mm and the distance between two cathode planes is also 2.1~mm, the wires are stretched over about 1~m. Belle II is building a cylindrical drift chamber very similar to what is needed for this experiment for which the space between wires is around 2.5~mm \cite{BelleIItdr}. Finally, a drift chamber with wires distant of 1~mm is being built for the small wheel of ATLAS at CERN \cite{ATLASChamber}. The cylindrical drift chamber proposed for our experiment is 400~mm long, we therefore considered that a 3~mm gap between wires would be a rather conservative goal. Optimization is envisaged in the future based on experience with prototypes. \\

However, the radial form of the detector does not allow for 90 degrees x-y wires in the chamber. So we use a stereo angle between wires to determine the coordinate along the beam axis ($z$). It will make it possible to use a very thin forward end-plate to reduce multiple scattering of the outgoing high-energy electrons. A rough evaluation of the weight due to about 2600 400~mm long wiresis under 100~kg, which appears to be reasonnable for a composite endplate. \\

Our drift chamber cells are composed of one sense wire made of gold plated tungsten surrounded by field wires, however the presence of the 5~T magnetic field complicates the field lines. We have studied several structures with MAGBOLTZ \cite{Magboltz}. The best compromise between the number of wires and regular drift lines has been obtained for the cell shown in Fig.~\ref{fig:drift_cell}. The sense wire is surrounded by 6 field wires placed equidistant from it. Two field wires are placed at the same radius as the field wire, two are at a smaller radius and two are at a larger one. The distance between the sense and field wires is constant and equal to 2.8~mm. Two adjacent cells share the field wire placed between them. They are 6 layers of cells at increasing radius from the target. \\

\begin{figure}
  \begin{center}
    \includegraphics[angle=0, width=0.5\textwidth]{./../Detector/fig-chap2/HEISOE.pdf}
    \caption{Drift lines simulated using MAGBOLTZ \cite{Magboltz} for one sense wire (at the center) surrounded by 6 field wires. The two electric field lines leaving the cell are removed when adjusting the voltages on the wires.}
    \label{fig:drift_cell}
  \end{center}
\end{figure}

The simulation code MAGBOLTZ is calculating the drift speed and drift paths of the electrons (Fig.~\ref{fig:drift_cell}). With a moderate electric field, the drift speed is around 10~microns/ns, the maximum drift time expected is thus 300~ns (over 3~mm). Assuming a conservative 15~ns time resolution, the spatial resolution will be around 200~microns. 

The maximum rates are expected for protons to be around 5~MHz for $2.10^{34}$~cm$^{-2}$s$^{-1}$, for an integration time of 300~ns and considering 6 layers of sense wires where two wires are about 3~mm distant, the total occupancy is expected around 2.5~\% which should be reasonnable to keep a good tracking. When running with the helium 4 target, it is not necessary to detect the protons, the rate can then be highly reduced by increasing the threshold, thus making the chamber blind to protons. \\

For preamplification of the signals, the first possibility investigated is to use the same preamplifier as the one developped for the CLAS inner calorimeter and improved for the Heavy Photon Search \cite{HPS} experiment installed in the Hall B. Depending on the gain in the drift chamber and the number of primary ionisations, it is possible to tune the gain of the preamplifier to adapt it to the needs of this experiment. More studies will be needed to evaluate how the gains of the chamber and the preamplifier can be tuned to ensure a noise that allows to discriminate clearly electrons from protons. The time resolution of HPS has been shown to around 1.6~ns for all crystals (Fig.~\ref{fig:HPSTreso}) which is much better than our requirements. For this task we will also use the experience from the Belle II small drift chamber electronics, that has similar requirements.

\begin{figure}
  \begin{center}
    \includegraphics[angle=0, width=0.5\textwidth]{./../Detector/fig-chap2/timing_fit_gauss_3pol}
    \caption{Typical time resolution of a crystal for HPS calorimeter.}
    \label{fig:HPSTreso}
  \end{center}
\end{figure}

\subsection{The scintillator array}

The scintillator array will have two main purposes. With its very good time resolution it will be included in the trigger to help reducing the background. It will also contribute to the particle identification. Its length should not exceed 400~mm making possible to reach a time resolution of 100~ps. It must also have a segmentation that will allow a good matching with the track reconstructed in the drift chamber. To improve our identification capabilities (see below) we also plan to segment the scientillators in several layers. Additional simulations will be done to optimize the geometry and thickness of the scintillators, but we show below that the simple proposed solution will already work for our experiment.

\section{Reconstruction scheme} \label{sec:sim}

Particle identification requires stopping the recoiling nucleus, measuring its time of arrival and energy deposition. The scheme used here is to measure the time of arrival of the particle in the scintillators along with its radius reconstructed by the tracking algorithm of the drift chamber. Using this method $\alpha$ and $^2$H have similar behavior. It is however easy to distinguish them using the energy desposited and the time of arrival in the scintillators. Moreover, if several layers of scintillators are used, the $\alpha$ will let a signal only in the first, making it even easier to distinguish them from the other particles. 

The track obtained using a helix fit is used to determine the coordinates of the vertex point and the transverse momentum of the particle. The energy deposited in the solid detector can also used to determine the kinetic energy of the nucleus. The feasibility and precision of the proposed vertex reconstruction and particle identification scheme are investigated with GEANT4 simulation.

The simulation of the recoil detector has been implemented with the full geometry and material specifications. It includes a 5~Tesla homogeneous solenoid field. The entire detector is filled with a mixture of He(90\%) and iC$_4$H$_{10}$(10\%) at 1~atm which is a very light gas. It allows to reduce energy loss and limit the energy deposition by minimum ionizing particles.  \\

In the current study all recoil species are generated with the same distributions: flat in momentum from threshold up to 40~MeV ($\sim$~250~MeV/c) for protons and about 25~MeV for other particles; isotropic angular coverage; flat distribution in $z$-vertex; and a radial vertex coordinate smeared around the beam line center by a Gaussian distribution of sigma equal the beam expectation radius (0.200 mm). \\

With the requirement that the particle reaches the scintillator and with a 40~cm length limit, there is a smoothly varying acceptance when averaged over the $z$-vertex position. This is shown from simulation in Fig.~\ref{fig:acceptance} for the lightest and heaviest recoil nuclei. However, this is a conservative estimate, as with tracking only information a more elaborate PID scheme may be able to accommodate a larger acceptance for lower energy recoils.\\

\begin{figure}[ht!]
    \begin{center}
        \includegraphics[width=0.45\textwidth]{./../Detector/fig-chap2/Bare_3atm_1atm_Proton_Acceptance}
        \includegraphics[width=0.45\textwidth]{./../Detector/fig-chap2/Bare_3atm_1atm_Alpha_Acceptance}
        \caption{Simulated recoil detector acceptance, for protons (left) and $^4$He (right), requiring energy deposition in the scintillators array. \label{fig:acceptance}}
    \end{center}
\end{figure}

First, the tracking capabilities of the recoil detector are investigated assuming a spatial resolutions of 200~$\mu$m for the drift chamber. The wires are strung in the $z$-direction with a stereo angle of 10$^\circ$. For particles stoped in the scintillators, the resulting difference between generated and reconstructed variables from simulation is shown in Fig.~\ref{fig:tracking} for $^4$He particles. The momentum for protons and $^4$He was also reconstructed (Fig.~\ref{fig:presolution}) from the radius of the helix assuming a uniform 5~T field. From these plots, it is clear that the resolutions required are fullfilled. \\

\begin{figure}[ht!]
    \begin{center}
        \includegraphics[height=4.5cm, width=0.32\textwidth]{./../Detector/fig-chap2/Bare_3atm_1atm_Alpha_ResoZ}
        \includegraphics[height=4.5cm, width=0.32\textwidth]{./../Detector/fig-chap2/Bare_3atm_1atm_Alpha_ResoPhi}
        \includegraphics[height=4.5cm, width=0.32\textwidth]{./../Detector/fig-chap2/Bare_3atm_1atm_Alpha_ResoTh}
        \caption{Simulated resolutions, integrated over $z$ for $^4$He, of the $z$-vertex (in mm) and the polar and azimuthal angles (in rad) for the lowest energy regime when the recoil track reaches the scintillator. \label{fig:tracking}}
    \end{center}
\end{figure}
\begin{figure}[tbp]
    \begin{center}
        \includegraphics[width=0.45\textwidth]{./../Detector/fig-chap2/Bare_3atm_1atm_Proton_ResoPt}%proton__pdp_sigma__regime1_small.png}
        \includegraphics[width=0.45\textwidth]{./../Detector/fig-chap2/Bare_3atm_1atm_Alpha_ResoPt}%alpha__pdp_sigma__regime1_small.png}
        \caption{Simulated momentum resolutions for proton (left) and $^4$He (right) integrated over $z$, when the recoil track reaches the scintillators array.\label{fig:presolution}}
    \end{center}
\end{figure}

Next, the particle identification scheme is investigated. Assuming a conservative 100~ps resolution of the scintillator and an energy resolution of 10\%, clean separation of three of the five nuclei is shown in Fig.~\ref{fig:SIMtof} which represents the time of arrival in the scintillator as a function of the reconstructed radius in the drift chamber. While not directly a concern for our experiment, we note that using energy deposition versus the radius reconstructed in the drift chamber one can separate $^2$H from $\alpha$. Also, the comparison of the energy deposited versus the radius and the angle $\theta$ can be used to improve the results.\\

\begin{figure}[ht!]
    \begin{center}
        \includegraphics[width=0.7\textwidth]{./../Detector/fig-chap2/Bare_3atm_1atm_RvsTime_named}
        \caption{Simulated time of flight at the scintillator versus the reconstructed radius in the drift chamber. The bottom band correspond to proton, next band is the $^3$He nuclei, $^2$H and $\alpha$ are overlapping in the third band, the uppermost band is $^3$H.\label{fig:SIMtof}}
    \end{center}
\end{figure}

We finally found particle identification efficiency of 99\% for protons, 95\% for $^3$He and 98\% for $^3$H. This conservative analysis suggests that the proposed reconstruction and particle identification schemes for this design are quite promising. Studies, using both software and prototyping, are ongoing to determine the optimal detector parameters to minimize the detection threshold while maximizing particle identification efficiency. The resolutions presented above have been implemented in a fast Monte Carlo used in the next section to evaluate our discovery potential.


