\addcontentsline{toc}{chapter}{Abstract}
{\large\textbf{Abstract:}}

We propose to measure semi-inclusive deep inelastic scattering from light nuclei ($^2$H and $^4$He). The detection of the low energy recoil (p, $^3$H and $^3$He), in addition to the scattered electron, will provide unique information about the nature of the EMC effect and its dependence on the bound nucleon virtuality. In particular, the proposed experiment will provide stringent tests leading to clear differentiation between the many models describing the EMC effect.

The proposed measurements will use the 11~GeV electron beam on thin gaseous targets and the upgraded CEBAF Large Acceptance Spectrometer (CLAS12) with a new recoil detector. The latter will be installed inside the solenoid magnet instead of the CLAS12 Silicon Vertex Tracker (SVT). The scattered electrons will be detected in CLAS12, while the fragments of the nuclear target will be detected in the recoil detector. 

The experiment requires a total of 70 days of running time, assuming a maximum of 200 nA beam current. We will need 30 days at 11 GeV for the measurements on $^4$He, 35 days at 11 GeV for the measurements on deuterium and 5 days at various energies for calibration runs using hydrogen and helium targets. Half of this time (xxx days) can be shared with the already approved BoNuS12 experiment (E12-06-113) when using the deuterium target. The beam time for calibration and with helium target (xxx days) can be shared with the proposed helium DVCS experiments.


\newpage

